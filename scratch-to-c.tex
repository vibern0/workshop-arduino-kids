\documentclass{article}
%code
\usepackage{listings}
%images
\usepackage{graphicx}
%quotes
\usepackage{csquotes}
%encoding
%--------------------------------------
\usepackage[utf8]{inputenc}
\usepackage[T1]{fontenc}
%--------------------------------------
 
%Portuguese-specific commands
%--------------------------------------
\usepackage[portuguese]{babel}
%--------------------------------------
 
%Hyphenation rules
%--------------------------------------
\usepackage{hyphenat}
\hyphenation{mate-mática recu-perar}
%--------------------------------------

\begin{document}

\title{No mundo da eletrónica e programação}
\author{Ética e Deontologia - Grupo xf}

\maketitle

\begin{abstract}
Obrigado a quem fez o almoço hoje.
\end{abstract}

\section{Introdução}
Bem-vindos!

\section{Controlo}
O controlo baseia-se em condições de acesso que desencadeiam determinadas ações, como por exemplo só quando estiver a pressionar o botão, vai abrir as cortinas da sala, no entanto se tiver a pressionar esse botão não vai iniciar o processo, vai simplesmente deixar que este continue a decorrer. Se eventualmente o botão deixar de ser pressionado então é necessário parar.

\subsection{espere}
\begin{tabular}{ c c }
    \includegraphics{imgs/controlers/espere} & delay(1000) \\
\end{tabular}

\section{Operadores}
Os operações efetuam uma determinada operação que normalmente são efetuadas por duas "coisas" do mesmo tipo, ou seja, entre dois numeros ou entre duas (ou mais) letras.

\section{Variáveis}
As variáveis servem exencialmente para guardar dados durante a execução de um programa.

\section{Mais arduino}
%outras coisas que o arduino tem que o scratch nao tem

\section{Módulos}
%os módulos mais conhecidos

\section{Links úteis}
%

\section{Exemplos}
%exemplos em scartch e traduzidos para c



\section{Conclusão}
Conclusão aqui.

\end{document}
